\section{Problems}\label{sec:figures}
\subsection{Part I - Mathematical modelling}\label{subsec:part1}
\subsubsection{Problem 1}

In the following derivations, Newton's $2^{\text{nd}}$ law of motion for rotation is used:

\begin{equation}\label{eq:P1_N2}
    J \dot{\omega} = \sum \tau = \sum \left(r\cdot F\right)
\end{equation}
Additionally, the definitions
\begin{subequations}\label{eq:P1_Vd_Vs_definitions}
    \begin{align}
        V_d &= V_f - V_b\label{eq:P1_Vd_definition} \\
        V_s &= V_f + V_b\label{eq:P1_Vs_definition}
    \end{align}    
\end{subequations}
are introduced to simplify the derived equations. Lastly, it is assumed that the moments of inertia about the pitch, elevation and travel axis is, respectively

\begin{subequations}\label{eq:P1_moments_of_inertia}
    \begin{align}
        J_p &= 2m_pl_p^2 \label{eq:P1_moment_of_inertia_p} \\
        J_e &= m_cl_c^2 + 2m_pl_h^2 \label{eq:P1_moment_of_inertia_e} \\
        J_{\lambda} &= m_cl_c^2 + 2m_p(l_h^2 + l_p^2) \label{eq:P1_moment_of_inertia_lambda}
    \end{align}
\end{subequations}
and where all constants used are according to~\cref{fig:heli_dimensions}.

The equations of motion about the pitch axis can be computed by using~\cref{eq:P1_N2},~\cref{fig:heli} and \cref{eq:P1_forces_propellers}. The forces that generate a moment about the pitch axis are the forces generated by the motors, $F_f = K_f V_f$ and $F_b = -K_f V_b$, and their weight decomposed, $F_{g,f}=-m_pg\cos(p)$ and $F_{g,b}=m_pg\cos(p)$ according to~\cref{eq:P1_weight_m_p}, where the forces are signed according to~\cref{fig:heli}. Given the discussed considerations, the equation of motion for rotation about the pitch axis can be derived as follows:

\begin{align}\label{eq:P1_pitch_non-linear}
    J_p \ddot{p} &= l_p F_f - l_p F_b \nonumber \\
    J_p \ddot{p} &= l_p K_f V_f - l_p K_f V_b \nonumber \\
    J_p \ddot{p} &= l_p K_f (V_f - V_b) \nonumber \\
    J_p \ddot{p} &= L_1 V_d
\end{align}
where the definition
\begin{equation}\label{eq:P1_L_1}
    L_1 = l_p K_f
\end{equation}
is introduced.

For the elevation, the same procedure as for pitch is used by considering how the motor forces' components about the elevation axis are a function of $p$ and that the weight of the motors and the counterweight have arms that are functions of $e$. This yields:

\begin{align}\label{eq:P1_elevation_non-linear}
    J_e \ddot{e} &= g l_c m_c \cos{e} - 2 g l_h m_p \cos{e} + K_f l_h V_f \cos{p} + K_f l_h V_b \cos{p} \nonumber \\
    J_e \ddot{e} &= L_2 \cos{e} + L_3 V_s \cos{p}
\end{align}
where the definitions
\begin{equation}\label{eq:P1_L_2}
    L_2 = g (l_c m_c - 2 l_h m_p)
\end{equation}
and
\begin{equation}
    L_3 = K_f l_h
\end{equation}
are made.

Finally, to calculate travel, one needs to consider only the motor forces as they are the only considered forces able to generate moment about the travel axis. This component is a function of $p$. Additionally, the arm is a function of $e$. In total, this yields:

\begin{align}\label{eq:P1_travel_non-linear}
    J_\lambda \ddot{\lambda} &= -(K_f l_h V_f + K_f l_h V_b)\cos{e}\sin{p} \nonumber \\
    J_\lambda \ddot{\lambda} &= L_4 V_s \cos{e} \sin{p}
\end{align}
Here, the definition
\begin{equation}\label{eq:P1_L_4}
    L_4 = - K_f l_h
\end{equation}
is made.

The reason for the negative sign is because a positive pitch gives a negative travel rate, in accordance to~\cref{fig:heli}. 
Summarised
\begin{subequations}\label{eq:P1_equation_of_motion}
    \begin{align}
        \ddot{p} &= \frac{L_1 V_d}{J_p} 
        \label{eq:P1_p_motion} \\
        \ddot{e} &= \frac{L_2 \cos{e} + L_3 V_s \cos{p}}{J_e} 
        \label{eq:P1_e_motion} \\
        \ddot{\lambda} &= \frac{L_4 V_s \cos{e} \sin{p}}{J_\lambda}
        \label{eq:P1_lambda_motion}
    \end{align}
\end{subequations}