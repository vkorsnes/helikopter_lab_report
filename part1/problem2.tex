\subsubsection{Problem 2}
In this part we wish to linearise at the point $\mathbf{x^*} = [~p^*\quad e^*\quad\lambda^*~]^T =
[~0\quad0\quad0~]^T$ and $\mathbf{u^*} = [~V_s^*\quad V_d^*~]^T$ where
\begin{equation*}
    \begin{bmatrix}
        \tilde{p} \\
        \tilde{e} \\
        \tilde{\lambda} \\
    \end{bmatrix}
    =
    \begin{bmatrix}
        p \\
        e \\
        \lambda \\
    \end{bmatrix}
    -
    \begin{bmatrix}
        p^* \\
        e^* \\
        \lambda^* \\
    \end{bmatrix}
\quad\text{and}\quad
    \begin{bmatrix}
        \tilde{V_s} \\
        \tilde{V_d} 
    \end{bmatrix}
    =
    \begin{bmatrix}
        V_s \\
        V_d
    \end{bmatrix}
    -
    \begin{bmatrix}
        V_s^* \\
        V_d^*
    \end{bmatrix}
\end{equation*}
is a coordinate transformation introduced to simplify further analysis. The linearisation will be with the helicopter horizontal in both pitch and elevation, as well as no travel (see~\cref{fig:heli} to see how angles are defined). For the equilibrium point, $V_d^*$ and $V_s^*$ needs to be determined. An equilibrium point implies that $\dot{p}=\dot{e}=\dot{\lambda}=0$, which again implies that $\dot{\dot{p}}=\dot{\dot{e}}=\dot{\dot{\lambda}}=0$. Based on this and ~\cref{eq:P1_pitch_non-linear}, one can deduce that $V_d^*$ will be zero, as anything else would eventually give a $\dot{p} \neq 0$ as $t\to\infty$.

To derive $V_s^*$ it is possible to use~\cref{eq:P1_elevation_non-linear}:

\begin{align*}
     J_e \ddot{e} &= L_2 \cos{e} + L_3 V_s \cos{p}\bigg{|}_{\mathbf{x}=\mathbf{x^*}} \\
     \implies\quad J_e\cdot0 &= L_2\cdot1 + L_3V_s^*\cdot1 \\
     V_s^* &= -\frac{L_2}{L_3} \\
     V_s^* &= -\frac{g(l_c m_c - 2 l_h m_p)}{K_f l_h}
\end{align*}
To summarise:
\begin{subequations}
    \begin{align}
        V_s^* &= -\frac{g(l_c m_c - 2 l_h m_p)}{K_f l_h} \label{eq:P1p2_Vs_asterix} \\
        V_d^* &= 0 \label{eq:P1p2_Vd_asterix}
    \end{align}    
\end{subequations}
To simplify further analysis, the following coordinate transformation is introduced.
\begin{equation}
    \begin{bmatrix}
        \tilde{p} \\ \tilde{e} \\ \tilde{\lambda}
    \end{bmatrix}
    =
    \begin{bmatrix}
        p \\ e \\ \lambda
    \end{bmatrix}
    -
    \begin{bmatrix}
        p^* \\ e^* \\ \lambda^*
    \end{bmatrix}
    \quad \text{and} \quad
    \begin{bmatrix}
        \tilde{V}_s \\ \tilde{V}_d
    \end{bmatrix}
    =
    \begin{bmatrix}
        V_s \\ V_d
    \end{bmatrix}
    -
    \begin{bmatrix}
        V_s^* \\ V_d^*
    \end{bmatrix}
\end{equation}
Which gives the following motion equations.
\begin{equation}
    \begin{bmatrix}
        p \\ e \\ \lambda
    \end{bmatrix}
    =
    \begin{bmatrix}
        \tilde{p} \\ \tilde{e} \\ \tilde{\lambda}
    \end{bmatrix}
    \quad \text{and} \quad
    \begin{bmatrix}
        V_s \\ V_d
    \end{bmatrix}
    =
    \begin{bmatrix}
        \tilde{V}_s \\ \tilde{V}_d
    \end{bmatrix}
    +
    \begin{bmatrix}
        -\frac{L_2}{L_3} \\ 0
    \end{bmatrix}
\end{equation}
To linearise, we use the Jacobian and evaluate in our equilibrium
\begin{equation}
    a_{i,j} = \left. \frac{\partial f_i}{\partial x_j} \right 
    |_{\tilde{p} = \tilde{e} = \tilde{\lambda} = 0}
\end{equation}

Where $f_i$ represents the function from problem 1, \cref{eq:P1_equation_of_motion} and $x_j$ represents the states.
We define the following point $P = (\tilde{p} = 0, \tilde{e} = 0, \tilde{\lambda} = 0)$
From this the matrices become
\begin{equation}
    \begin{bmatrix}
        \td{p} \\ \tdd{p} \\ \td{e} \\ \tdd{e} \\
        \td{\lambda} \\ \tdd{\lambda}
    \end{bmatrix}
    =
    \setstackgap{L}{2.1\baselineskip}
    \fixTABwidth{T}
    \bracketMatrixstack{
        \lin{\td{p}}{\tilde{p}} &\lin{\td{p}}{\td{p}} 
        &\lin{\td{p}}{\tilde{e}} &\lin{\td{p}}{\td{e}} 
        &\lin{\td{p}}{\tilde{\lambda}} &\lin{\td{p}}{\td{\lambda}} \\
        \lin{\tdd{p}}{\tilde{p}} &\lin{\tdd{p}}{\td{p}} 
        &\lin{\tdd{p}}{\tilde{e}} &\lin{\tdd{p}}{\td{e}} 
        &\lin{\tdd{p}}{\tilde{\lambda}} &\lin{\tdd{p}}{\td{\lambda}} \\
        \lin{\td{e}}{\tilde{p}} &\lin{\td{e}}{\td{p}} 
        &\lin{\td{e}}{\tilde{e}} &\lin{\td{e}}{\td{e}} 
        &\lin{\td{e}}{\tilde{\lambda}} &\lin{\td{e}}{\td{\lambda}} \\
        \lin{\tdd{e}}{\tilde{p}} &\lin{\tdd{e}}{\td{p}} 
        &\lin{\tdd{e}}{\tilde{e}} &\lin{\tdd{e}}{\td{e}} 
        &\lin{\tdd{e}}{\tilde{\lambda}} &\lin{\tdd{e}}{\td{\lambda}} \\
        \lin{\td{\lambda}}{\tilde{p}} &\lin{\td{\lambda}}{\td{p}}
        &\lin{\td{\lambda}}{\tilde{e}} &\lin{\td{\lambda}}{\td{e}}
        &\lin{\td{\lambda}}{\tilde{\lambda}} &\lin{\td{\lambda}}{\td{\lambda}} \\
        \lin{\tdd{\lambda}}{\tilde{p}} &\lin{\tdd{\lambda}}{\td{p}}
        &\lin{\tdd{\lambda}}{\tilde{e}} &\lin{\tdd{\lambda}}{\td{e}}
        &\lin{\tdd{\lambda}}{\tilde{\lambda}} &\lin{\tdd{\lambda}}{\td{\lambda}}
    }
    \begin{bmatrix}
        \tilde{p} \\ \td{p} \\ \tilde{e} \\ \td{e} \\
        \tilde{\lambda} \\ \td{\lambda}
    \end{bmatrix}
    +
    \setstackgap{L}{2.1\baselineskip}
    \fixTABwidth{T}
    \bracketMatrixstack{
        \lin{\td{p}}{\tilde{V}_s} &\lin{\td{p}}{\tilde{V}_d}  \\
        \lin{\tdd{p}}{\tilde{V}_s} &\lin{\tdd{p}}{\tilde{V}_d}  \\
        \lin{\td{e}}{\tilde{V}_s} &\lin{\td{e}}{\tilde{V}_d}  \\
        \lin{\tdd{e}}{\tilde{V}_s} &\lin{\tdd{e}}{\tilde{V}_d}  \\
        \lin{\td{\lambda}}{\tilde{V}_s} &\lin{\td{\lambda}}{\tilde{V}_d} \\
        \lin{\tdd{\lambda}}{\tilde{V}_s} &\lin{\tdd{\lambda}}{\tilde{V}_d}
    }
    \begin{bmatrix}
        \tilde{V}_s \\ \tilde{V}_d
    \end{bmatrix}
\end{equation}
With values
\begin{equation}
    \begin{bmatrix}
        \td{p} \\ \tdd{p} \\ \td{e} \\ \td{e} \\
        \td{\lambda} \\ \td{\lambda}
    \end{bmatrix}
    =
    \setstackgap{L}{1.05\baselineskip}
    \fixTABwidth{T}
    \bracketMatrixstack{
        0   &1  &0  &0  &0  &0 \\
        0   &0  &0  &0  &0  &0 \\
        0   &0  &0  &1  &0  &0 \\
        0   &0  &0  &0  &0  &0 \\
        0   &0  &0  &0  &0  &1 \\
        -\frac{L_4 L_2}{J_\lambda L_3}  &0   &0   &0   &0   &0 
    }
    \begin{bmatrix}
        \tilde{p} \\ \td{p} \\ \tilde{e} \\ \td{e} \\
        \tilde{\lambda} \\ \td{\lambda}
    \end{bmatrix}
    +
    \setstackgap{L}{1.05\baselineskip}
    \fixTABwidth{T}
    \bracketMatrixstack{
        0                   &0               \\
        \frac{L_1}{J_p}     &0               \\
        0                   &0               \\
        0                   &\frac{L_3}{J_e} \\
        0                   &0               \\
        0                   &0              
    }
    \begin{bmatrix}
        \tilde{V}_s \\ \tilde{V}_d
    \end{bmatrix}
\end{equation}
This simplifies to
\begin{subequations}\label{eq:P1_linearised_equations_of_motion}
    \begin{align}
        \td{p} &= \td{p} \nonumber \\ 
        \tdd{p} &= K_1 \tilde{V}_s \label{eq:P1_linearised_equation_of_motion_p_ddot} \\
        \td{e} &= \td{e} \nonumber \\
        \tdd{e} &= K_2 \tilde{V}_d \label{eq:P1_linearised_equation_of_motion_e_ddot} \\
        \td{\lambda} &= \td{\lambda} \nonumber \\
        \tdd{\lambda} &= K_3 \tilde{p} \label{eq:P1_linearised_equation_of_motion_lambda_ddot}
    \end{align}
\end{subequations}    
Where
\begin{subequations}
    \begin{align}
        K_1 &= \frac{L_1}{J_p} \label{eq:P1_lin_K1} \\
        K_2 &= \frac{L_3}{J_e} \label{eq:P1_lin_K2} \\
        K_3 &= -\frac{L_4 L_2}{J_\lambda L_3} \label{eq:P1_lin_K3}
    \end{align}
\end{subequations}