\subsection{Part IV - State estimation}\label{subsec:part4}
\subsubsection{Problem 1}
The system given in \cref{eq:P1_linearised_equations_of_motion} can be expressed as a state space representation with the vectors
\begin{align}
    \textbf{x} = \begin{bmatrix}
        \tilde p        \\
        \td p           \\
        \tilde e        \\
        \td e           \\
        \tilde \lambda  \\
        \td \lambda
    \end{bmatrix}, \hspace{0.5cm}
    \mathbf{u} = \begin{bmatrix}
        \tilde V_s \\
        \tilde V_d
    \end{bmatrix} \hspace{0.5cm} \text{and} \hspace{0.5cm}
    \textbf{y} = \begin{bmatrix}
        \tilde p        \\
        \tilde e        \\
        \tilde \lambda  \\
    \end{bmatrix}.
\end{align}
The system can thus be expressed on the form
\begin{align}
    \mathbf{\dot x} &= \mathbf{Ax} + \mathbf{B} \\
    \mathbf{y} &= \mathbf{Cx},
\end{align}
with the matrices
\begin{equation}
    \setstackgap{L}{1.1\baselineskip}
    \fixTABwidth{T}
    \label{p4_sys_AB}
    \mathbf A = 
    \bracketMatrixstack{
		0   & 1 & 0 & 0 & 0 & 0 \\
		0   & 0 & 0 & 0 & 0 & 0 \\
		0   & 0 & 0 & 1 & 0 & 0 \\
		0   & 0 & 0 & 0 & 0 & 0 \\
		0   & 0 & 0 & 0 & 0 & 1 \\
		K_3 & 0 & 0 & 0 & 0 & 0 
	}
	\text{,}
	\hspace{0.5cm}
	\mathbf B = 
	\bracketMatrixstack{
	    0   & 0   \\
	    0   & K_1 \\
	    0   & 0 & \\
	    K_2 & 0   \\
	    0   & 0   \\
	    0   & 0
	}
\end{equation}
and
\begin{equation}
    \label{p4_sys_C}
    \mathbf C = 
    \begin{bmatrix}
        1   &   0   &   0   &   0   &   0   &   0 \\
        0   &   0   &   1   &   0   &   0   &   0 \\
        0   &   0   &   0   &   0   &   1   &   0
    \end{bmatrix}.
\end{equation}