\subsubsection{Problem 3}
\textbf{Observability Analysis}\\
The matrices 
\begin{equation}
    \mathbf{C}_{e\lambda} = \begin{bmatrix}
    0 & 0 & 1 & 0 & 0 & 0 \\
    0 & 0 & 0 & 0 & 1 & 0
    \end{bmatrix}
    \quad \text{and} \quad
    \mathbf{C}_{pe} = \begin{bmatrix}
    1 & 0 & 0 & 0 & 0 & 0 \\
    0 & 0 & 1 & 0 & 0 & 0
    \end{bmatrix}
\end{equation}
correspond to the measurement vectors
\begin{equation}
    \mathbf{y}_{e\lambda} = \begin{bmatrix}
        \tilde e \\
        \tilde \lambda
    \end{bmatrix}
    \quad \text{and} \quad
    \mathbf{y}_{pe} = \begin{bmatrix}
        \tilde p \\
        \tilde e 
    \end{bmatrix}
\end{equation}
respectively.\\
Since the observability matrix,
\begin{equation}
    \setstackgap{L}{1.1\baselineskip}
    \fixTABwidth{T}
    \mathbf{\mathcal{O}}_{e\lambda} = 
        \begin{bmatrix}
        \mathbf{C}_{e\lambda}               \\
        \mathbf{C}_{e\lambda}\mathbf{A}     \\
        \mathbf{C}_{e\lambda}\mathbf{A^2}   \\
        \mathbf{C}_{e\lambda}\mathbf{A^3}   \\
        \mathbf{C}_{e\lambda}\mathbf{A^4}   \\
    \end{bmatrix}
    = \bracketMatrixstack{
    0       & 0       & 1 & 0 & 0 & 0 \\
    0       & 0       & 0 & 0 & 1 & 0 \\
    0       & 0       & 0 & 1 & 0 & 0 \\
    0       & 0       & 0 & 0 & 0 & 1 \\
    0       & 0       & 0 & 0 & 0 & 0 \\
    -0.612  & 0       & 0 & 0 & 0 & 0 \\
    0       & 0       & 0 & 0 & 0 & 0 \\
    0       & -0.612  & 0 & 0 & 0 & 0 \\
    0       & 0       & 0 & 0 & 0 & 0 \\
    0       & 0       & 0 & 0 & 0 & 0 \\
    0       & 0       & 0 & 0 & 0 & 0 \\
    0       & 0       & 0 & 0 & 0 & 0 }
\end{equation}
has full column rank, the system is observable when measuring only $\tilde e$ and $\tilde \lambda$. On the other hand, when measuring only $\tilde p$ and $\tilde e$, the system is \textit{not} observable, as

\begin{equation}
    \mathbf{\mathcal{O}}_{pe} = 
        \begin{bmatrix}
        \mathbf{C}_{pe}               \\
        \mathbf{C}_{pe}\mathbf{A}     \\
        \mathbf{C}_{pe}\mathbf{A^2}   \\
        \mathbf{C}_{pe}\mathbf{A^3}   \\
        \mathbf{C}_{pe}\mathbf{A^4}   \\
    \end{bmatrix}
    = \begin{bmatrix}
    1 & 0 & 0 & 0 & 0 & 0 \\
    0 & 0 & 1 & 0 & 0 & 0 \\
    0 & 1 & 0 & 0 & 0 & 0 \\
    0 & 0 & 0 & 1 & 0 & 0 \\
    0 & 0 & 0 & 0 & 0 & 0 \\
    0 & 0 & 0 & 0 & 0 & 0 \\
    0 & 0 & 0 & 0 & 0 & 0 \\
    0 & 0 & 0 & 0 & 0 & 0 \\
    0 & 0 & 0 & 0 & 0 & 0 \\
    0 & 0 & 0 & 0 & 0 & 0 \\
    0 & 0 & 0 & 0 & 0 & 0 \\
    0 & 0 & 0 & 0 & 0 & 0 \\
    \end{bmatrix}
\end{equation}
does \textit{not} have full column rank.\\
This also makes intuitive sense, as we cannot expect to observe the travel, $\lambda$, from only knowing elevation $e$ and the pitch $p$. But if we do know the travel, the pitch can be obtained by taking the derivative.\\ 
\\
\textbf{The Baddest Observer in the West}\\
Tuning the observer based solely on the outputs $e$ and $\lambda$ proved to be much harder than when the pitch, $p$ was also part of the measurements. The erroneous estimation is most clear in the pitch rate estimation as shown in figure (WE DONT HAVE A GOOD FIGURE WAT DO). This is partly because the pitch has the non-linear correlation with elevation and travel given in equations \eqref{eq:P1_elevation_non-linear} and \eqref{eq:P1_travel_non-linear}, but the observer is based on the linearised equations given in \eqref{eq:P1_linearised_equations_of_motion}. As a result, the estimation error becomes more prevalent the further away from the linearisation point the state sways. This error gets amplified further when taking the derivative, since a high frequency noise signal such as $\sin(\omega t)$, where $\omega$ is large, has the high amplitude derivative $\omega \cos(\omega t)$. 
    