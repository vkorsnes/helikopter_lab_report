\section{Problems}\label{sec:problems}
\subsection{Part IV}
\subsubsection{Problem 1}
The system given in the equations (TODO: ADD REFERENCE) can be expressed as a state space equation 
with the vectors
\begin{align}
    \textbf{x} = \begin{bmatrix}
        \tilde p        \\
        \td p           \\
        \tilde e        \\
        \td e           \\
        \tilde \lambda  \\
        \td \lambda
    \end{bmatrix}, \hspace{0.5cm}
    \mathbf{u} = \begin{bmatrix}
        \tilde V_s \\
        \tilde V_d
    \end{bmatrix} \hspace{0.5cm} \text{and} \hspace{0.5cm}
    \textbf{y} = \begin{bmatrix}
        \tilde p        \\
        \tilde e        \\
        \tilde \lambda  \\
    \end{bmatrix}
\end{align}.
The equation can thus be expressed on the form
\begin{align}
    \mathbf{\dot x} &= \mathbf{Ax} + \mathbf{B} \\
    \mathbf{y} &= \mathbf{Cx},
\end{align}
with the matrices
\begin{equation}
    \setstackgap{L}{1.1\baselineskip}
    \fixTABwidth{T}
    \label{p4_sys_AB}
    \mathbf A = 
    \parenMatrixstack{
		0   & 1 & 0 & 0 & 0 & 0 \\
		0   & 0 & 0 & 0 & 0 & 0 \\
		0   & 0 & 0 & 1 & 0 & 0 \\
		0   & 0 & 0 & 0 & 0 & 0 \\
		0   & 0 & 0 & 0 & 0 & 1 \\
		K_3 & 0 & 0 & 0 & 0 & 0 
	}
	\text{,}
	\hspace{0.5cm}
	\mathbf B = 
	\parenMatrixstack{
	    0   & 0   \\
	    0   & K_1 \\
	    0   & 0 & \\
	    K_2 & 0   \\
	    0   & 0   \\
	    0   & 0
	}
\end{equation}
and
\begin{equation}
    \label{p4_sys_C}
    \mathbf C = 
    \begin{pmatrix}
        1   &   0   &   0   &   0   &   0   &   0 \\
        0   &   0   &   1   &   0   &   0   &   0 \\
        0   &   0   &   0   &   0   &   1   &   0
    \end{pmatrix}
\end{equation}.
\subsubsection{Problem 2}
From the system matrix in \eqref{p4_sys_AB} and the output matrix in \eqref{p4_sys_C}, the observability matrix can be found to be
\begin{equation}
    \setstackgap{L}{1.1\baselineskip}
    \fixTABwidth{T}
    \mathbf{\mathcal{O}} = 
        \begin{pmatrix}
        \mathbf{C}      \\
        \mathbf{CA}     \\
        \mathbf{CA^2}   \\
        \mathbf{CA^3}   \\
        \mathbf{CA^4}   \\
        \mathbf{CA^5}   \\
    \end{pmatrix}
    =
    \parenMatrixstack{
    1       & 0       & 0 & 0 & 0 & 0 \\
    0       & 0       & 1 & 0 & 0 & 0 \\
    0       & 0       & 0 & 0 & 1 & 0 \\
    0       & 1       & 0 & 1 & 0 & 0 \\
    0       & 0       & 0 & 0 & 0 & 1 \\
    0       & 0       & 0 & 0 & 0 & 0 \\
    0       & 0       & 0 & 0 & 0 & 0 \\
    -0.6117 & 0       & 0 & 0 & 0 & 0 \\
    0       & 0       & 0 & 0 & 0 & 0 \\
    0       & 0       & 0 & 0 & 0 & 0 \\
    0       & -0.6117 & 0 & 0 & 0 & 0 \\
    0       & 0       & 0 & 0 & 0 & 0 \\
    0       & 0       & 0 & 0 & 0 & 0 \\
    0       & 0       & 0 & 0 & 0 & 0 \\
    0       & 0       & 0 & 0 & 0 & 0 \\
    0       & 0       & 0 & 0 & 0 & 0 \\
    0       & 0       & 0 & 0 & 0 & 0 }
    \text{.}
\end{equation}
We note that $\mathbf{\mathcal{O}}$ has full column rank, which means the system is observable.\\
\\
We wish to control the system using an estimated state $\mathbf{\hat{x}}$. A \textit{linear observer} (or closed-loop estimator) is then
\begin{equation}
    \mathbf{\dot{\hat{x}}} = \mathbf{A\hat{x}} + \mathbf{Bu} + \mathbf{L} (\mathbf{y} - \mathbf{C\hat{x}})
\end{equation}
where $\mathbf A$, $\mathbf B$ and $\mathbf C$ are as before and $\mathbf L$ is the \textit{observer gain matrix}.

\subsubsection{Problem 3}
The matrices 
\begin{equation}
    \mathbf{C}_{e\lambda} = \begin{pmatrix}
    0 & 0 & 1 & 0 & 0 & 0 \\
    0 & 0 & 0 & 0 & 1 & 0
    \end{pmatrix}
    \quad \text{and} \quad
    \mathbf{C}_{pe} = \begin{pmatrix}
    1 & 0 & 0 & 0 & 0 & 0 \\
    0 & 0 & 1 & 0 & 0 & 0
    \end{pmatrix}
\end{equation}
correspond to the measurement vectors
\begin{equation}
    \mathbf{y}_{e\lambda} = \begin{bmatrix}
        \tilde e \\
        \tilde \lambda
    \end{bmatrix}
    \quad \text{and} \quad
    \mathbf{y}_{pe} = \begin{bmatrix}
        \tilde p \\
        \tilde e 
    \end{bmatrix}
\end{equation}
respectively.\\
Since
\begin{equation}
    \setstackgap{L}{1.1\baselineskip}
    \fixTABwidth{T}
    \mathbf{\mathcal{O}}_{e\lambda} = 
        \begin{pmatrix}
        \mathbf{C}_{e\lambda}               \\
        \mathbf{C}_{e\lambda}\mathbf{A}     \\
        \mathbf{C}_{e\lambda}\mathbf{A^2}   \\
        \mathbf{C}_{e\lambda}\mathbf{A^3}   \\
        \mathbf{C}_{e\lambda}\mathbf{A^4}   \\
    \end{pmatrix}
    = \parenMatrixstack{
    0       & 0       & 1 & 0 & 0 & 0 \\
    0       & 0       & 0 & 0 & 1 & 0 \\
    0       & 0       & 0 & 1 & 0 & 0 \\
    0       & 0       & 0 & 0 & 0 & 1 \\
    0       & 0       & 0 & 0 & 0 & 0 \\
    -0.6117 & 0       & 0 & 0 & 0 & 0 \\
    0       & 0       & 0 & 0 & 0 & 0 \\
    0       & -0.6117 & 0 & 0 & 0 & 0 \\
    0       & 0       & 0 & 0 & 0 & 0 \\
    0       & 0       & 0 & 0 & 0 & 0 \\
    0       & 0       & 0 & 0 & 0 & 0 \\
    0       & 0       & 0 & 0 & 0 & 0 }
\end{equation}
has full column rank, the system is observable when measuring only $\tilde e$ and $\tilde \lambda$. On the other hand, when measuring only $\tilde p$ and $\tilde e$, the system is \textit{not} observable, as

\begin{equation}
    \mathbf{\mathcal{O}}_{pe} = 
        \begin{pmatrix}
        \mathbf{C}_{pe}               \\
        \mathbf{C}_{pe}\mathbf{A}     \\
        \mathbf{C}_{pe}\mathbf{A^2}   \\
        \mathbf{C}_{pe}\mathbf{A^3}   \\
        \mathbf{C}_{pe}\mathbf{A^4}   \\
    \end{pmatrix}
    = \begin{pmatrix}
    1 & 0 & 0 & 0 & 0 & 0 \\
    0 & 0 & 1 & 0 & 0 & 0 \\
    0 & 1 & 0 & 0 & 0 & 0 \\
    0 & 0 & 0 & 1 & 0 & 0 \\
    0 & 0 & 0 & 0 & 0 & 0 \\
    0 & 0 & 0 & 0 & 0 & 0 \\
    0 & 0 & 0 & 0 & 0 & 0 \\
    0 & 0 & 0 & 0 & 0 & 0 \\
    0 & 0 & 0 & 0 & 0 & 0 \\
    0 & 0 & 0 & 0 & 0 & 0 \\
    0 & 0 & 0 & 0 & 0 & 0 \\
    0 & 0 & 0 & 0 & 0 & 0 \\
    \end{pmatrix}
\end{equation}
does \textit{not} have full column rank.\\
This also makes intuitive sense, as we cannot expect to observe the travel, $\lambda$, from only knowing elevation $e$ and the pitch $p$. But if we do know the travel, the pitch can be obtained by taking the derivative.  

